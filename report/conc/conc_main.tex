\chapter{Conclusions}

From the theoretical analysis made, the benchmarks performed and the
results obtained, the main conclusion in this work is that the
lattice-Boltzmann method may indeed be used to model electrokinetic
systems. In \cite{lbm-wang}, a coupling of the same equations as in
this work is done. Also a lattice-Boltzmann approach for solving the
equations is proposed. There are however several differences between
what is done here and in \cite{lbm-wang}. First, different equilibrium
distributions are used for the Nernst-Planck and Poisson's
equations. In \cite{lbm-wang}, a Navier-Stokes-like equilibrium
including quadratic terms\footnote{One of the ``quadratic terms'' is
  actually missing the square, this must be a typo. Otherwise the
  concentration is not obtained as the zeroth moment as stated. It
  also does not agree with the reference used for motivating the
  equilibrium distribution.} are used for the Nernst-Planck equation
and for the Poisson's equation different weights are used. Second,
different implementations of the boundary conditions are used. Also,
the no-flux boundary condition used in the Nernst-Planck equation is
integrated into the fluid domain and rewritten as a Dirichlet
condition. This is not very accurate, since the boundary condition
only apply for the boundary and not in the interior of the
domain. Further, some results are contradictory to the system
modelled, e.g. it is stated that a 1:1 solution is modelled even
though the computed, charge density is strictly positive. It must
therefore be some source of positive or leak of negative ions present,
a guess is that it is due to the incorrect boundary treatment for the
ion flux at the channel walls.

The lack of scientific results on the lattice-Boltzmann method or
results which sometimes seem a bit off, is a sign of the youth of the
method and a rather great limitation when working with it. There are
few ``school books'' with detailed explanations. Instead, scientific
papers where all details sometimes are not written out explicitly
have to be addressed.

When browsing work on modelling of electrokinetics, usually the
Poisson-Boltzmann method is utilised. Sometimes even without
reflection of whether it is applicable or not. From this work it is
concluded that as the thickness of the double layer gets comparable to
the dimensions of the system considered or if advection is present,
the Poisson-Boltzmann approach may not be an accurate model. The
underlying assumptions used in the derivation must be studied and
assured to be fulfilled in the system under consideration.

In situations where the Poisson-Boltzmann approach fails, the more
general method proposed in this work may be used instead. Few results
of the more complicated systems modelled in this work was found and it
is therefore difficult to determine a qualitative and quantitative
correctness. Experimental results for this kind of system is very
difficult to obtain, the detailed measurement of a velocity
field in a system is not simple to determine. This is also the main
reason why a computational approach is so much desired.

To get an efficient implementation of the method, topics from
high-performance computing must be considered. The distribution
function must be organised in memory to allow for good locality in the
computation. Also unnecessary branches and data dependence should be
avoided in the innermost loop. It is important to use a modern and
uptodate compiler and it may be a good idea to tell the compiler to
optimise the code as much as possible. The LBM algorithm is also very
well suited for parallelisation. This is due to that there is no data
dependence between nodes in the collision step.
