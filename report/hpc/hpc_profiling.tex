\section{Profiling}
In the development phase of optimising a code, a most valuable tool is
a profiler. Basically it is a program that allows for analysing the
program as it runs, e.g. to give information of which instructions are
executed, if the CPU stalls or if and when cache misses are
present. This information is also connected to the source code which
gives easy access for the developer to correct eventual performance
issues.

A popular profiler used on UNIX systems is \texttt{gprof}
\cite{gprof}. In this work, it is used to determine which routines
that most contribute to the total execution time. It is in those
routines that a code optimisation is most profitable. In order to be
able to profile with \texttt{gprof}, the code must be compiled with
additional compiler flags (\texttt{-pg} with \texttt{gcc}). This
profilable program will run slower than a program compiled without
profiling flags, it is thus important to not forget to compile the
``finished'' code without profiling flags. Below is an example of some
output given by \texttt{gprof} when profiling the Navier-Stokes LBM
implementation:

\begin{verbatim}
  %   cumulative   self                         
 time   seconds   seconds    calls  name    
 40.98      1.61     1.61 40000000  BGKNS::get1moment(int, int, double*)
 30.54      2.81     1.20 40000000  BGKNS::fEq(double, double*, double*)
 10.69      3.23     0.42 40000000  BGKNS::get0moment(int, int)
 10.18      3.63     0.40     1000  StreamD2Q9::stream()
 ...
\end{verbatim}
for brevity, only the output of most time consuming routines are
shown. These four routines stands together for about 90$\;$\% of the
total execution time. The main part of this time is for computing the
equilibrium distribution which includes computing the zeroth and first
moments of the distribution function. The reason why the first moment
is four times as time consuming as the zeroth moment, is that the
implementation is for a general lattice using the lattice vectors
instead of just hardcoding it for the D2Q9 lattice which would have
given a computation two times as time consuming as the zeroth moment.
This is an example of when less general approach may give a
performance boost on the cost of, in this case, to have to define
separate routines for each lattice rather than having one general
routine.

\subsection{Memory specific profiling}
Even though \texttt{gprof} is a good tool in many profiling
situations, there may be cases where more low level information such
as cache misses or certain instruction counts are desired. Fortunately
there exist other more specialised tools to deal with these
situations. One popular set of tools for memory profiling/debugging is
\texttt{valgrind} \cite{valgrind}. Basically it works by simulating a
virtual CPU and e.g. memory accesses and cache misses are counted as a
program is run on this virtual CPU. In this work the
\texttt{cachegrind} tool of \texttt{valgrind} is used to count cache
misses, here is an example output for the Navier-Stokes
implementation:

\begin{verbatim} 
 I   refs:      33,986,419,147
 I1  misses:             1,828
 LLi misses:             1,771
 I1  miss rate:           0.00%
 LLi miss rate:           0.00%
 
 D   refs:      17,284,324,380  (12,564,330,224 rd   + 4,719,994,156 wr)
 D1  misses:       364,260,481  (   354,685,981 rd   +     9,574,500 wr)
 LLd misses:       175,534,108  (   165,973,036 rd   +     9,561,072 wr)
 D1  miss rate:            2.1% (           2.8%     +           0.2%  )
 LLd miss rate:            1.0% (           1.3%     +           0.2%  )
 
 LL refs:          364,262,309  (   354,687,809 rd   +     9,574,500 wr)
 LL misses:        175,535,879  (   165,974,807 rd   +     9,561,072 wr)
 LL miss rate:             0.3% (           0.3%     +           0.2%  )
\end{verbatim}
where \texttt{I}, \texttt{D} and \texttt{LL} is the instruction, L1
and last level (L3 in this particular case) caches respectively. The
grid is chosen large enough to keep the size of the $f$ array much
larger ($\sim$ 300 MiB) than the 8 MiB L3 cache. We see that about 1\%
of the data memory lookups result in cache misses. A more thorough
investigation using \texttt{cg\_annotate} of in what routines the
cache misses arise, gives that about two thirds are in the streaming
step and about one third in the collision step. This behaviour is in
accordance with the memory model chosen, see section
\ref{sec:hpc:loc_lbm}.

\nomenclature{LL}{Last level (cache)}
\nomenclature{UNIX}{A computer operating system}
