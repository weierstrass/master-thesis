\section{Statistical background}\label{sec:lbm:stat}
Consider one litre of air. At NTP, the volume will contain in the
order of $10^{22}$ molecules. In order to model this system
microscopically, 6 variables per molecule will be needed to describe
the microstate of the system. Just to store the state of the system in
a computer would require more space than the estimated size of the
whole world wide web times one million \cite{wolfram-alpha-web}. Thus,
for these kind of systems, the microscopic approach is somewhat
impractical.

For that reason, statistical approaches have been developed for these
kind of problems. A fundamental quantity used for describing the
system is a continuous probability density distribution, $f$. This
distribution may be regarded as an average over the
microstates. Consider a volume of $d^3\x d^3\mathbf{p}$ in phase
space, the number of molecules, $dN$ in this volume is then given
through the density distribution, $f$ as

\begin{equation}
dN(\x, \p, t) = f(\x, \p, t)\drm^3\x \drm^3\p.
\end{equation}
Thus $f(\x, \p, t)$ is a measure for the number of particles at
location $\x$ with momentum $\p$ at time $t$. Macroscopic variables is
obtained by summing, e.g. the particle density, $n$, is obtained from

\begin{equation}
n(\x, t) = \int f(\x, \p, t) \drm^3\p
\end{equation}
and the macroscopic momentum density of the system is determined by

\begin{equation}
rho(\x, t) \ubf(\x, t) = \int \p f(\x, \p, t) \drm^3\p
\end{equation}
where $\rho = m n$ is the mass density. Multiplying $f$ by a power $k$ of
$\p$ or $\mathbf{v} = \p/m$ and integrating is often referred to as
taking the $k$:th moment of $f$ and is a term that will be used
throughout this chapter.

In the late 19th century, Boltzmann developed a model for the time
evolution of $f$. To do this he had to make several
assumptions. First, only collisions between two particles are
considered, this makes the equation mostly applicable to dilute
gases. Second, the two particles colliding are assumed to be
uncorrelated before the collision. Third, external forces are assumed
not to affect the collisions \cite{wolf-gladrow}. The equation is
named after its father to the Boltzmann (transport) equation and
reads:

\begin{equation}\label{eq:lbm:boltzmann-eq}
\partial_t f + \frac{\p}{m} \cdot \nabla_{\x}f + \frac{\F}{m} \cdot
\nabla_{\mathbf{v}}f = Q(f, f)
\end{equation}
where $f$ is the distribution function for a single species collection
of particles of mass $m$, $\F$ is external forces, $\p$ is momentum,
$\nabla_{\x}$ and $\nabla_{\mathbf{v}}$ are the gradients in location and
velocity space respectively. The right-hand side contains the so
called collision term which in the general case is expressed as an
integral. 

\begin{equation}
Q(f, f) = \int \int ...
\end{equation}

This integral states how the distribution function changes after a two
particle collision. However, the structure of this integral is in most
physical situations too complicated to be used directly. Therefore, a
number of simplifications have been proposed during the years.

When designing these approximations, at least two main properties of
the collision integral must be kept. \cite{wolf-gladrow}

\begin{enumerate}
  \item The same quantities that is conserved under collisions in the
    collsion integral must also be conserved in the approximation.
  \item Boltzmann's H-theorem must be fulfilled for the
    approximated collision operator.
\end{enumerate}

Without being to specific, the H-theorem states that the entropy
computed from $f$ is always increasing with time and that the maximum
entropy is obtained for a so called Maxwellian distribution in
momentum/velocity space. Boltzmann used an other quantity denoted by H
closely related to entropy, thus the name of the theorem. The
Maxwellian distribution in two dimensions that $f$ tends towards is
given by

\begin{equation}\label{eq:lbm:maxwell}
f^{(M)}(\x, \mathbf{v}, t) = n \left ( \frac{m}{2 \pi k_B T} \right )
\exp{\left( -\frac{m}{2 k_B T} (\mathbf{v} - \ubf)^2\right)}
\end{equation} 
where $\ubf(\x, t)$ is the mean velocity of the particles in the system and
$n(\x, t)$ is the number of particles at location $\x$. In section
\ref{sec:lbm:col}, one of the most widely used approximations of the
collision integral will be presented.
