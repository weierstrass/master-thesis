\section{The Boltzmann transport equation}
Consider one litre of air. At NTP, the volume will contain in the
order of $10^{22}$ molecules. In order to model this system
microscopically, 6 variables per molecule will be needed to describe
the microstate of the system. Just to store the state of the system in
a computer would require more space than the estimated size of the
whole world wide web times one million \cite{wolfram-alpha-web}. Thus,
for these kind of systems, the microscopic approach is somewhat
impractical.

Fortunately, statistical approaches have been developed for these kind
of problems. A fundamental quantity used for describing the system is
a continuous probability density distribution. Consider a volume of
$d^3\x d^3\mathbf{p}$ in phase space, the number of molecules, $dN$ is
then given through the density distribution, $f$ as

\begin{equation}
dN(\x, \p, t) = f(\x, \p, t)d^3\x d^3\p.
\end{equation}
Thus $f(\x, \p, t)$ is a measure for the number of particles at
location $\x$ with momentum $\p$ at time $t$. 

In the late 19th century, Boltzmann developed a model for the time
evolution of $f$. To do this he had to make several
assumptions. First, only collisions between two particles are
considered, this makes the equation mostly applicable to dilute
gases. Second, the two particles colliding are assumed to be
uncorrelated before the collision. Third, external forces are assumed
not to affect the collisions \cite{wolf-gladrow}. The equation is
named after its father to the Boltzmann (transport) equation and reads:

\begin{equation}\label{eq:lbm:boltzmann-eq}
\partial_t f + \frac{\p}{m} \cdot \nabla_{\x}f + \frac{\F}{m} \cdot
\nabla_{\ubf}f = Q(f, f)
\end{equation}
where $f$ is the distribution function for a single species collection
of particles of mass $m$, $\F$ is external forces, $\p$ is momentum,
$\nabla_{\x}$ and $\nabla_{\ubf}$ are the gradients in location and
velocity space respectively. The rights hand side contains the so
called collision term which in the general case is expressed as an
integral. However, in applications, it is usually approximated by some
simpler expression.
