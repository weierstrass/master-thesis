\section{Boundary conditions}\label{sec:lbm:bound}
In all physical situations, when solving a differential equations on a
domain, conditions for what is happening on the boundary of the domain
must be specified. So far, only the update rule for $\fii$ on the
interior of this domain have been treated. In this section we will
also define rules for the boundaries of the domain. The nodes in the
interior and the boundary will be referred to as \emph{interior nodes}
and \emph{boundary nodes} respectively. Typically, a boundary
condition in a macroscopic variable, e.g. velocity, is specified from
the physical problem. This condition must be translated into a
condition for the distribution function, $\fii$, on the statistical
level. In this section some, to this work, useful boundary conditions
will be formulated and discussed.

\subsection{Bounce-back boundaries}
accuracy, e.g. second order accurate if placed between node planes...

\subsection{Slip boundaries}

\subsection{he-zou, constant density/velocity}

\subsection{Maybe something on non-local boundary conditions}
