\section{Physical and lattice units}

A physical system may in its physical units contain quantities whose
numerical values may differ a lot from each other. It is therefore,
from a numerical point of view, a very bad idea and in the case with
lattice-Boltzmann it will in general not work to just solve the
equation in physical units. The equation is therefore scaled to
dimensionless form by introducing characteristic quantities. For
example a quantity A is scaled by A$_0$ to a dimensionless quantity
A$'$ through

\begin{equation}
\mathrm{A}' = \mathrm{A}/\mathrm{A_0} 
\end{equation}
Usually the dimensionless quantities are wanted to be of order one.

In the case with Navier-Stokes equation, three characteristic
quantities are needed to put it on dimensionless form. I.e. one for
the density ($\rho_0$), velocity ($u_0$) and length ($\ell_0$). The
non-dimensional incompressible N-S will contain one dimensionless
parameter, i.e. the Reynolds number $\Rerm = \rho_0 u_0 \ell_0 / \mu$.

For the advection-diffusion equation, in addition to a characteristic
length and velocity, a characteristic concentration ($\C_0$) is
needed. One dimensionless parameter arise when non-dimensionalising the
equation, i.e. the Peclet number, $\Pe = u_0\ell_0/D$ where $D$ is the
diffusion coefficient. 

Poisson's equation is non-dimensionalised by using a length and a
characteristic voltage ($V_0$). The RHS then gets multiplied by factor
$\ell_0^2/V_0$ in the non-dimensional form.

When the non-dimensional form of the equations are determined, the
domain is discretised. The non-dimensional length of the system is now
typically 1 and $\delta_x = 1/N$ where $N$ is the number of lattice
cells used. From the diffusive scaling used, $\delta_t = \delta_x^2$
is usually a good choice.  
