\section{The BGK collision operator}\label{sec:lbm:col}
The collision term in the LBE is the main ingredient in what
determines the physics of the system that is being modelled. Here the
desired interaction of the pseudo particles is stated. In section
\ref{sec:lbm:stat}, two necessary properties to approximations of the
full collision integral was stated.

One of the simplest collision operators that fulfil conditions (1)
and (2) in section \ref{sec:lbm:stat} is the BGK operator (BGK from
its creators: Bhatnagar, Gross and Krook). It was proposed in 1954 and
is today one of the most commonly used collision operators both in the
case of the lattice-Boltzmann and the continuous Boltzmann
equation. It is based on the principle of relaxing $f$ towards a
Maxwellian distribution. The relaxation is also performed in such a
way that the collision invariants are preserved.  In the discrete case,
eq. \eqref{eq:lbm:lbe} the operator is given by:

\begin{equation}\label{eq:lbm:bgk}
\Omega_{ij} = \Omega_i = -\omega \left[ f_i(\x, t) - f_i^{(eq)}(\x, t)
  \right]
\end{equation}
where $\omega$ is a parameter determining the relaxation rate and
$\feq$ should be an equilibrium distribution that makes sure that the
necessary conditions are fulfilled. In the discrete case, a truncated
expansion of eq. \eqref{eq:lbm:maxwell} is typically used
\cite{wolf-gladrow}. This gives for instance


\begin{equation}
\feq = w_i\rho \left [ 1 + \frac{\ci \cdot \ubf}{c_s^2} +
  \frac{(\ci \cdot \ubf)^2}{2c_s^4} - \frac{\ubf^2}{2c_s^2} \right]
\end{equation}
where $w_i$ is a lattice specific weight, $\rho$ is the zeroth moment
of $\fii$ and $\ci$ is a unit velocity in the discretised velocity
space.

The BGK operator is due to its simplicity both when it comes to
theoretical treatment and implementation a popular choice. However in
some physical situations, e.g. multi-phase or high Reynolds-number
flows, more sophisticated alternatives are required
\cite{wolf-gladrow}. Throughout this work, the BGK operator will be
used.

\nomenclature{BGK}{Relaxation type collision operator (Bhatnagar,
  Gross and Krook)}
