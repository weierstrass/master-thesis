\section{The collision operator}\label{sec:lbm:col}
The collision term in the LBE is basically what determines the physics
of the system that is beeing modelled. Here the desired interaction of
the pseudo particles is stated. In the continuous case with the
Boltzmann equation, Boltzmann formulated a two particle (this was an
assumption of his model) collision integral. However, the structure of
this integral is in most physical situations too complicated to be
used directly. Therefore, a number of simplifications have been
proposed during the years. 

\todo{consider writing the  integral explicitly...}

When designing these approximations, at least two main properties of
the collision integral must be kept. \cite{wolf-gladrow}

\begin{enumerate}
  \item The same quantities that is conserved under collisions in the
    collsion integral must also be conserved in the approximation.
  \item Boltzmann's H-theorem must also be fulfilled for the
    approximated collision operator.
\end{enumerate}

Without being to specific, the H-theorem states that the entropy
computed from $f$ is always increasing with time and that the maximum
entropy is obtained for a so called Maxwellian distribution. Boltzmann
used an other quantity denoted by H closely related to entropy, thus
the name of the theorem. The Maxwellian distribution that $f$ tends
towards is given by

\begin{equation}\label{eq:lbm:maxwell}
f^{(M)}(\x, \mathbf{v}, t) = n \left ( \frac{m}{2 \pi k_B T} \right )
\exp{\left( -\frac{m}{2 k_B T} (\mathbf{v} - \ubf)^2\right)}
\end{equation} 
where $\ubf$ is the mean velocity of the particles in the
system. 

\subsection{The BGK operator}
One of the simplest collision operators that fullfil conditions (1)
and (2) is the BGK operator (BGK from its creators: Bhatnagar, Gross
and Krook). It was proposed in 1954 and is today one of the most
comonly used collision operators both in the case of the
Lattice-Boltzmann and the continuous Boltzmann equation. It is based
on the principle of relaxing $f$ towards a Maxwellian
distribution. The relaxation is also performed in such a way that the
collision invariants is preserved.  In the discrete case,
eq. \eqref{eq:lbm:lbe} the operator is given by:

\begin{equation}\label{eq:lbm:bgk}
\Omega_i = -\omega \left[ f_i(\x, t) - f_i^{(eq)}(\x, t) \right]
\end{equation}
where $\omega$ is a parameter determining the relaxation rate and
$\feq$ should be of ``Maxwellian type'' int order to satisfy condition
(2). In the discrete case, eq. \eqref{eq:lbm:maxwell} is taylor
expanded around $\ubf = 0$ which gives an expression of the type

\begin{equation}
\feq = w_i\rho \left [ 1 + ..........\right]
\end{equation}
where $\rho$ is the zeroth moment of $fi$.

\subsection{Other choices}
