\section{Asymptotic analysis}\label{sec:lbm:asym}
Methods from asymptotic analysis will, in this section, be used to
investigate the macroscopic limit of the general LBE. More detailed and
specific analyses for the three different equations considered will be
presented in sections \ref{sec:lbm:asym_np}, \ref{sec:lbm:asym_ns} and
\ref{sec:lbm:asym_pe}. 

Asymptotic analysis is basically about describing mathematical objects
in some limit, e.g. how a function behaves for large or small values
of some parameter. Consider for example the series $S_{\ep}$:

\begin{equation}
S_{\ep} = a^{(0)} + a^{(1)}\epsilon + a^{(2)}\epsilon^2 +
a^{(3)}\epsilon^3 + \mathcal{O}(\epsilon^4)
\end{equation}
It is clear that for sufficiently small values of $\epsilon$, the
terms of higher order is negligible to those of lower order and the
series may be truncated at some point and still be a good
approximation of $S_{\ep}$. For example if $\ep$ is small, then is
$S_{\ep} \approx a^{(0)}$ and we say that if $a^{(1)} \neq 0$ that
this approximation is of first order accuracy.

There are different approaches to go from the discrete LBE to a
continuous macroscopic equation. The most frequently applied one to
obtain the Navier-Stokes equations is the Chapman-Enskog method
\cite{junk-boundary}, which will reproduce the compressible
equations. An other method, often employed by M. Junk and his
associates e.g. in \cite{junk-asym}, is a method based on regular
asymptotic expansions, this is also the method that will be utilised
in this work and will in the case of Navier-Stokes reproduce the
incompressible equations. A brief discussion of the differences
between the Chapman-Enskog and the regular expansion approaches will
be carried out at the end of this chapter.

The basic idea behind the analysis is to expand the distribution
function $\fii$ in some small parameter, $\epsilon$. Also this
parameter will be related to the spatial and time scales. The
macroscopic limit is obtained by taking the Taylor expansion of the
discrete LBE and comparing terms of equal order in
$\epsilon$. Together with the fact that certain quantities is
invariant under collisions, macroscopic differential equations is
obtained. Now follows the part of the analysis which is common for the
three equations, the more equation specific analysis is carried out in
sections \ref{sec:lbm:asym_np}, \ref{sec:lbm:asym_ns} and
\ref{sec:lbm:asym_pe} respectively.

\subsection{Motivation of the choice of expansion parameter}
A most desired property of the expansion parameter is that it should
be a small and dimensionless number. If the lattice is dense enough with
respect to the characteristic length scale of the system, a suitable
choice is the Knudsen number, $\ep$, which is defined as the ratio of
the mean free path, $\delta_x$, and the characteristic length of the
system under consideration, $\ell_0$, i.e. $\epsilon = \delta_x
/\ell_o$. To be able to perform the asymptotic analysis we must also
relate the time scale to this parameter. From the fact that the
lattice speed $c_s = \delta_x/\delta_t$ and by introducing a
characteristic speed, $u_o = \ell_0/t_0$, we have

\begin{equation}\label{eq:lbm:rel}
\epsilon = \frac{\delta_x}{\ell_0} = \frac{c_s}{u_0}\frac{\delta_t}{t_0}
\end{equation}
It is now clear that what determines the relation between the
timescale and the parameter $\epsilon$ is the ratio of the
characteristic speed and the lattice speed which is usually referred
to as the Mach number, Ma. In our particular case we will operate in
the incompressible limit, i.e. Ma $\ll$ 1 and a suitable choice is a
small number, thus Ma = $\epsilon$ is chosen \cite{junk-boundary}. The
discretisation of the space and time step is then related through

\begin{equation}
\delta_x'^2 = \delta_t' = \epsilon^2
\end{equation}
where the primes denote dimensionless variables. This particular
scaling is usually referred to as diffusive scaling.

\subsection{Expanding the LBE}
The LBE, eq. \eqref{eq:lbm:lbe}, with dimensionless variables and the
BGK collision operator reads:

\begin{equation}\label{eq:lbm:nodim_lbe}
f_i(\x' + \epsilon \cbf_i', t' + \epsilon^2) - f_i(\x', t') = -\omega \left[
  f_i(\x', t') - f_i^{(eq)}(\x', t') \right].
\end{equation} 
The primes denoting dimensionless variables will, for readability
reasons, from hereon be dropped. If nothing else is stated we always
consider dimensionless variables.

To obtain a differential equation, the difference equation in
eq. \eqref{eq:lbm:nodim_lbe} is Taylor expanded, which gives

\begin{equation}\label{eq:lbm:taylor_lbe}
\ep(\pd\fii) + \ep^2 (\partial_t\fii + (\pd \fii)^2/2 ) + \ep^3
(\partial_t (\pd\fii) + (\pd\fii)^3/6) + \bigO{\ep^4} = 
-\omega \left[
  f_i - f_i^{(eq)} \right]
\end{equation}
Expanding also $\fii$ and $\feq$ in the parameter $\epsilon$:

\begin{equation}\label{eq:lbm:fi_exp}
\fii = \fie{0} + \ep\fie{1} + \ep^2\fie{2} + \ep^3\fie{3} + \bigO{\ep^4}
\end{equation}

\begin{equation}\label{eq:lbm:fi_eq_exp}
\feq = \feqe{0} + \ep\feqe{1} + \ep^2\feqe{2} + \ep^3\feqe{3} +
\bigO{\ep^4}
\end{equation}
and inserting these expressions into eq. \eqref{eq:lbm:taylor_lbe}
gives an equation with terms of varying orders of $\ep$. Separating
this equation in equations of common orders allows for an analysis of
what happens at different scales of $\epsilon$. For the four leading
orders in $\ep$ we have:

\begin{equation}\label{eq:lbm:ep0}
\ep^0:\;\; 0 = -\omega \left[
  \fie{0} - \feqe{0} \right],
\end{equation}

\begin{equation}\label{eq:lbm:ep1}
\ep^1:\;\; \pd\fie{0} = -\omega \left[
  \fie{1} - \feqe{1} \right],
\end{equation}

\begin{equation}\label{eq:lbm:ep2}
\ep^2:\;\; \pd\fie{1} + \partial_t\fie{0} + (\pd \fie{0})^2/2 =
-\omega \left[ \fie{2} - \feqe{2} \right]
\end{equation}
and
\begin{equation}\label{eq:lbm:ep3}
\ep^3:\;\; \pd\fie{2} + \partial_t\fie{1} + (\pd \fie{1})^2/2 +
\partial_t (\pd\fie{0}) + (\pd\fie{0})^3/6 = -\omega \left[ \fie{3} -
  \feqe{3} \right].
\end{equation}

The idea is now that for an equation of a particular order in $\ep$,
use collision invariants and eliminate unknown $\fie{n}$:s by using
equations of lower order in $\ep$. This will in the end result in
differential equations for macroscopic variables, given by moments of
the $\fii$:s.
