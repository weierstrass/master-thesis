\section{LBM for the incompressible Navier-Stokes}\label{sec:lbm:ns}
The most frequent use of the LBM is to solve the Navier-Stokes
equations. In this work only the incompressible case,
eqs. \eqref{eq:et:ns_incompressible} and \eqref{eq:et:ns_mom} will be
considered. The LBM may however be used to reproduce weak
compressibility \cite{wolf-gladrow}. The incorporation of forces will
be treated in section \ref{sec:lbm:forces}. We recall the
incompressible Navier-Stokes equations without external forces present
from chapter \ref{sec:et} as:

\begin{equation}\label{eq:lbm:ns_inc}
 \nabla \cdot \ubf = 0
\end{equation}
and

\begin{equation}\label{eq:lbm:ns_mom}
\rhorm \left (\dfrac{\partial \ubf}{\partial t} +
  \ubf \cdot \nabla \ubf 
  \right ) = - \nabla \Prm  + \rhorm \nu \nabla^2 \ubf
\end{equation}
where $\nu$ is the kinematic viscosity related through the
dynamic viscosity, $\mu$, by

\begin{equation}
\mu = \rhorm \nu.
\end{equation}

A LBM will now be formulated for eqs. \eqref{eq:lbm:ns_inc} and
\eqref{eq:lbm:ns_mom}. The LBE with a BGK collision operator is given
by

\begin{equation}
f_i(\x + \cbf_i\delta_t, t + \delta_t) - f_i(\x, t) = -\omega \left[ f_i(\x, t) - f_i^{(eq)}(\x, t) \right]
\end{equation}
where

\begin{equation}\label{eq:lbm:ns_eq}
\feq = w_i\rho \left [ 1 + \frac{\ci \cdot \ubf}{c_s^2} +
  \frac{(\ci \cdot \ubf)^2}{2c_s^4} - \frac{\ubf^2}{2c_s^2} \right]
\end{equation}
where $w_i$ are the weights in eq. \eqref{eq:lbm:weights}.

The density, $\rhorm$, and the mass flux, $\rhorm \ubf$, is determined
from $\fii$ by taking the zeroth and first moments respectively:

\begin{equation}\label{eq:lbm:rho_mom}
\rhorm = \sum_i \fii
\end{equation}
and
\begin{equation}\label{eq:lbm:j_mom}
\rhorm \ubf = \sum_i \fii \ci .
\end{equation}
The kinematic viscosity is related to the relaxation parameter,
$\omega$

\begin{equation}\label{eq:lbm:nu}
\nu = c_s^2\left( \frac{1}{\omega} - \frac{1}{2} \right).
\end{equation}

\subsection{Asymptotic analysis}\label{sec:lbm:asym_ns}
Partially based on \cite{junk-asym}, an asymptotic analysis of the
above suggested method will be performed. In most literature
available, when reproducing the Navier-Stokes equations in the
macroscopic limit, a Chapman-Enskog expansion is performed. Note that
this is \emph{not} what is done here.

From the expansion of $\fii$ in eq. \eqref{eq:lbm:fi_exp} follows the
expansion of the macroscopic mass and velocity as

\begin{equation}\label{eq:lbm:ns_rho_exp}
\rho = \rhoe{0} + \ep\rhoe{1} + \ep^2\rhoe{2} + \ep^3\rhoe{3} +
\bigO{\ep^4}
\end{equation} 
and

\begin{equation}
\ubf = \ue{0} + \ep\ue{1} + \ep^2\ue{2} + \ep^3\ue{3} + \bigO{\ep^4}.
\end{equation}

These expansions are plugged into the equilibrium distribution in
eq. \eqref{eq:lbm:ns_eq} and from the equation of order zero in $\ep$,
eq. \eqref{eq:lbm:ep0} gives:

\begin{equation}
  \fie{0} = w_i \rhoe{0}
\end{equation}
Here $\ue{0}$ has been assumed to be zero by the same argumentation as
in the Nernst-Planck analysis, section
\ref{sec:lbm:asym_np}. Continuing to the equation of order 1 in $\ep$,
eq. \eqref{eq:lbm:ep1} and taking the first moment gives

\begin{equation}\label{eq:lbm:ns_rhoe0}
\nabla \rhoe{0} = 0.
\end{equation}
Further, by using this fact that $\rhoe{0}$ is constant in space and
that

\begin{equation}
\fie{1} = - \frac{1}{\omega} \pd\fie{0} + \feqe{1}
\end{equation}
where

\begin{equation}
\feqe{1} = w_i\left[ \rhoe{1} + \rhoe{0}\frac{\ci \cdot \ue{1}}{c_s^2}
  \right]
\end{equation}
gives when taking the zeroth moment of the equation of order two in $\ep$,
eq. \eqref{eq:lbm:ep2}

\begin{equation}
\partial_t \rhoe{0} + \nabla \cdot (\rhoe{0} \ue{1}) = 0.
\end{equation}
This equation states the conservation of mass for $\rhoe{0}$. Since
we are considering only systems in the incompressible limit,
$\rhoe{0}$ will be assumed to also be constant in time and we have

\begin{equation}\label{eq:lbm:ns_nodivu}
\nabla \cdot \ue{1} = 0.
\end{equation}
Taking the first moment of eq. \ref{eq:lbm:ep2} gives 

\begin{equation}\label{eq:lbm:ns_rhoe1}
\nabla \rhoe{1} = 0.
\end{equation}
It can be showed \cite{junk-asym} that if $\rhoe{1}$ is initialised
properly, then it does not change in time. Therefore, if $\rhoe{1}$ is
initialised to zero, it will also remain zero for all time.

As we now continue to the equation of order three in $\ep$, things
will get a bit more technical as we will make use of the fourth order lattice isotropy, eq. \eqref{eq:lbm:i4}. The zeroth moment of the equation gives

\begin{equation}
\nabla \cdot \ue{2} = 0.
\end{equation}
which may be used to show that $\ue{2} = 0$ \cite{junk-asym}, it will
not be shown here. Instead we continue by taking the first moment of
the equation, as it gets a bit more technical now, some intermediate
steps in the calculation will explicitly be written out. First from
the equation of order two in $\ep$ we have that

\begin{equation}
\fie{2} = -\frac{1}{\omega} \pd \fie{1} - \frac{1}{\omega} \partial_t
\fie{0} - \frac{1}{2\omega} (\pd \fie{0})^2 + \feqe{2}
\end{equation}
where

\begin{equation}
\feqe{2} = w_i \left[ \rhoe{2} + \rhoe{0} \frac{(\ci \cdot
    \ue{1})^2}{2c_s^4} - \rhoe{0} \frac{\left(\ue{1}\right)^2}{2c_s^2} \right]
\end{equation}
where the fact that $\ue{2} = 0$ have been used. Inserting the
expression for $\fie{2}$ into eq. \eqref{eq:lbm:ep3} gives, in index
notation

\begin{equation}\label{eq:lbm:foo1}
\begin{aligned}
\cc{\alpha}\cc{\beta}\parti{\beta} \left[ - \frac{\rhoe{0}}{\omega
    c_s^2} \cc{\gamma}\parti{\gamma}\cc{\delta} \uc{\delta} +
  w_i\rhoe{2} +
  \frac{\rhoe{0}w_i}{2c_s^4}\left(\cc{\gamma}\cc{\delta}\uc{\gamma}\uc{\delta}
  - (\uc{})^2 \right) \right] + \\ + \partial_tw_i\rhoe{0}\uc{\alpha} +
\frac{\rhoe{0}}{2 c_s^2}
\cc{\alpha}\cc{\beta}\parti{\beta}\cc{\gamma}\parti{\gamma}\cc{\delta}
= 0
\end{aligned}
\end{equation}
For brevity, the terms that in the end will equal zero have been
omitted. The indices $\alpha, \beta, \gamma$ and $\delta$ may take the
$x$ or $y$ component respectively. Summing eq. \eqref{eq:lbm:foo1}
over the directional index $i$ and by using the isotropity properties
of the lattice, eq. \eqref{eq:lbm:i0} gives:

\begin{equation}\label{eq:lbm:dino}
\begin{aligned}
c_s^2\rhoe{0}\left( \frac{1}{2} - \frac{1}{\omega}
\right)\parti{\beta} \left( \deltasec \right) + c_s^2\parti{\beta}\dab
\rhoe{2} +\\ + \frac{\rhoe{0}}{2}\left(( \deltasec) \uc{\gamma}\uc{\delta} -
\dd{\alpha}{\beta}(\uc{})^2 \right) + \rhoe{0}\partial_t \uc{\alpha} = 0
\end{aligned}
\end{equation}
On the form it is written now, it is not totally clear whether the
above equation is the incompressible Navier-Stokes momentum equation
or not. A simplification is in order and for brevity only the
simplification of one term is shown here.

By noting that $(\uc{})^2$ may be written as
$\dd{\gamma}{\delta}\uc{\gamma}\uc{\delta}$ we have

\begin{equation}
\begin{aligned}
 \parti{\beta}\left[ (\deltasec) \uc{\gamma}\uc{\delta} - \dd{\alpha}{\beta}(\uc{})^2\right] =
  \\ =\parti{\beta}\left[(\dd{\alpha}{\beta}\dd{\gamma}{\delta}+
  \dd{\alpha}{\gamma}\dd{\beta}{\delta})\uc{\gamma}\uc{\delta}\right]
\end{aligned}
\end{equation}
and by summing over $\gamma$ and $\delta$ the above expression reduces
to:

\begin{equation}
\parti{\beta} \left[ \uc{\alpha}\uc{\beta} + \uc{\beta}\uc{\alpha}
  \right] = 2\parti{\beta}\uc{\alpha}\uc{\beta} 
\end{equation}
and from incompressibility ($\parti{\beta}\uc{\beta} = 0$),
eq. \eqref{eq:lbm:ns_nodivu} we end up with

\begin{equation}
2\uc{\beta}\parti{\beta}\uc{\alpha}.
\end{equation}
By analogous simplification of the remaining terms in
eq. \eqref{eq:lbm:dino} we get

\begin{equation}\label{eq:lbm:ns_mom_index}
\rhoe{0} \left( \partial_t \uc{\alpha} +
\uc{\beta}\parti{\beta}\uc{\alpha} \right) = -\parti{\beta}(c_s^2
\rhoe{2}) + \rhoe{0}\nu \parti{\beta} \left( \parti{\beta}\uc{\alpha} +
\parti{\alpha}\uc{\beta} \right)
\end{equation}
where $\nu$ is defined as in eq. \eqref{eq:lbm:nu}, and if the
quantity $c_s^2\rhoe{2}$ is interpreted as the pressure, the above
equation is indeed the incompressible Navier-Stokes momentum equation,
eq. \eqref{eq:lbm:ns_mom}. It is satisfied by $\ue{1}$ in the
expansion of $\ubf$, since $\ue{2} = 0$, the error will be of order
$\ep^3$ and we say that the LB formulation is of second order accuracy
in velocity. $\rhoe{0}$ satisfies the mass equation and since
$\rhoe{1} = 0$ we say that the method is second order in density as
well \cite{junk-asym}. Since the pressure is included as a term in the
density expansion, problems with keeping the incompressibility might
arise if large enough pressures are present.

\subsection{Forcing schemes}\label{sec:lbm:forces}
Several methods have been proposed to add the forcing term in
eq. \eqref{eq:et:ns_mom} to the LBE. In \cite{krafczyk} five of these
methods have been compared. To briefly summarise the article; for
single phase flows, the methods achieve comparable. 

Due to its simplicity, a method formulated by Shan and Chen
\cite{shan-chen} was used in this work. It consists of modifying the
equilibrium distribution by computing it with the modified velocity

\begin{equation}
\ubf^{(eq)} = \frac{1}{\rho}\left(\sum_i \fii \ci +
\frac{\F\delta_t}{\omega} \right)
\end{equation}
where $\F$ is the external force. The physical velocity, $\ubf$, is
computed by

\begin{equation}
\ubf = \frac{1}{\rho}\left(\sum_i \fii \ci +
\frac{\F\delta_t}{2} \right).
\end{equation} 
