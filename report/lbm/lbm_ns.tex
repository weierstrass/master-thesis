\section{LBM for the incompressible Navier-Stokes}\label{sec:lbm:ns}
The most frequent use of the LBM is to solve the Navier-Stokes
equations. In this work only the incompressible case,
eqs. \eqref{eq:et:ns_incompressible} and \eqref{eq:et:ns_mom} will be
considered. The LBM may however be used to reproduce weak
compressibility \cite{wolf-gladrow}. The incorporation of forces will
be treated in section \ref{sec:lbm:forces}. We recall the
incompressible Navier-Stokes equations without external forces present
from chapter \ref{sec:et} as:

\begin{equation}\label{eq:lbm:ns_inc}
 \nabla \cdot \ubf = 0
\end{equation}
and

\begin{equation}\label{eq:lbm:ns_mom}
\rhorm \left (\dfrac{\partial \ubf}{\partial t} +
  \ubf \cdot \nabla \ubf 
  \right ) = - \nabla \Prm  + \rhorm \nu \nabla^2 \ubf
\end{equation}
where $\nu$ is the kinematic viscosity related through the
dynamic viscosity, $\mu$, by

\begin{equation}
\mu = \rhorm \nu.
\end{equation}

A LBM will now be formulated for eqs. \eqref{eq:lbm:ns_inc} and
\eqref{eq:lbm:ns_mom}. The LBE with a BGK collision operator is given
by

\begin{equation}
f_i(\x + \cbf_i\delta_t, t + \delta_t) - f_i(\x, t) = -\omega \left[ f_i(\x, t) - f_i^{(eq)}(\x, t) \right]
\end{equation}
where

\begin{equation}\label{eq:lbm:ns_eq}
\feq = w_i\rho \left [ 1 + \frac{\ci \cdot \ubf}{c_s^2} +
  \frac{(\ci \cdot \ubf)^2}{2c_s^4} - \frac{\ubf^2}{2c_s^2} \right]
\end{equation}
where $w_i$ are the weights in eq. \eqref{eq:lbm:weights}.

The density, $\rhorm$, and the mass flux, $\rhorm \ubf$, is determined
from $\fii$ by taking the zeroth and first moments respectively:

\begin{equation}\label{eq:lbm:rho_mom}
\rhorm = \sum_i \fii
\end{equation}
and
\begin{equation}\label{eq:lbm:j_mom}
\rhorm \ubf = \sum_i \fii \ci .
\end{equation}
The kinematic viscosity is related to the relaxation parameter,
$\omega$

\begin{equation}
\nu = c_s^2\left( \frac{1}{\omega} - \frac{1}{2} \right).
\end{equation}

\subsection{Asymptotic analysis}\label{sec:lbm:asym_ns}
Partially based on \cite{junk-asym}, an asymptotic analysis of the
above suggested method will be performed. In most literature
available, when reproducing the Navier-Stokes equations in the
macroscopic limit, a Chapman-Enskog expansion is performed. Note that
this is \emph{not} what is used here.

From the expansion of $\fii$ in eq. \eqref{eq:lbm:fi_exp} follows the
expansion of the macroscopic mass and velocity as

\begin{equation}\label{eq:lbm:ns_rho_exp}
\rho = \rhoe{0} + \ep\rhoe{1} + \ep^2\rhoe{2} + \ep^3\rhoe{3} +
\bigO{\ep^4}
\end{equation} 
and

\begin{equation}
\ubf = \ue{0} + \ep\ue{1} + \ep^2\ue{2} + \ep^3\ue{3} + \bigO{\ep^4}.
\end{equation}

These expansions are plugged into the equilibrium distribution in
eq. \eqref{eq:lbm:ns_eq} and from the equation of order zero in $\ep$,
eq. \eqref{eq:lbm:ep0}section gives:

\begin{equation}
  \fie{0} = w_i \rhoe{0}
\end{equation}
Here $\ue{0}$ has been assumed to be zero by the same argumentation as
in the Nernst-Planck analysis, section
\ref{sec:lbm:asym_np}. Continuing to the equation of order 1 in $\ep$,
eq. \eqref{eq:lbm:ep1} and taking the first moment gives

\begin{equation}
\nabla \rhoe{0} = 0
\end{equation}



\subsection{Forcing schemes}\label{sec:lbm:forces}
how to add a ''forcing'' term in the method.

