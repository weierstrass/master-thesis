\chapter{The lattice-Boltzmann method}
Rather than modelling on a macroscopic or microscopic scale, the
lattice-Boltzmann method operates at a scale in between those, often
referred to as a mesoscopic scale.

short intro of this section
micro macro meso... new method..

\section{Historical overview}
With the introduction of electronic calculating machines came also a
new possibility of tackling problems. New fields of computational
science was born and methods for solving both new traditional problems
were developed. 

The idea of using a discrete and simplified version of the
Boltzmann-equation dates back to the mid 60's \cite{scholarpedia-lbm}
with an experimental attempt to model simple gas dynamics. However, at
the time, this kind of statistical computational approaches was not
considered a serious alternative for the modelling of more
sophisticated fluid behaviour. It was first in the mid 80's when
Frisch, Hasslacher and Pomeau showed that a lattice automaton, with a
lattice of certain symmetry and that conserved mass and momentum in
the collisions, reproduced the Navier-Stokes equations in a
macroscopic limit. It was with this paper and the always increasing
computational power that made the idea of fluid modelling on a
mesoscopic scale a serious research topic.

The lattice gas automata (LGA) approach suffered from some flaws,
e.g. that the boolean nature of the method introduced statistical
noise and that lack of symmetry in the lattice made the advection
non-isotropic. The statistical noise was usually dealt with by
averaging which resulted in a coarsed domain and the advection issue
was handled by introducing lattices of higher symmetry. As the flaws
of the LGA approach was dealt with one after one, the method evolved
into what we today know as the lattice-Boltzmann method, with the
crucial refinement of using continuous distributions over boolean
variables. \cite{wolf-gladrow}

\section{The Boltzmann transport equation}
The continuous lbe is bte.. distribution function


\section{Basic idea of the LBM}
As previously noted, the lattice-Boltzmann is a mesoscopic
method. This means that the modelling is neither done on a molecular
level nor by direct solving the macroscopic equations involved. The
aim, in the most situations with the lattice-Boltzmann is indeed to
solve some macroscopic equation but not direct. Instead a statistical
model is used with various mesoscopic variables that, in some limit,
reproduces the macroscopic variables. It is also possible to ensure
that these variables fulfil a certain macroscopic equation by using a
certain scheme.

Basically the lattice-Boltzmann method solves a discretised version of
eq. \eqref{eq:lbm:boltzmann-eq} for the distribution functions and
then macroscopic quantities are determined from the distribution
functions. Both the spatial positions and the velocity space is
discretised allowing the distributions to ``sit'' only at certain
positions and to stream to neighbouring locations only in the
directions specified by the discretised velocity space. Usually in two
dimensions the velocity space is discretised into 9 distinct
velocities, more about the choice of lattice is discussed in section
\ref{sec:lbm:lattice}. In this case 9 distribution functions are
needed per node which might correspond to one or two macroscopic
variables but is indeed fewer than the number of variables needed for
a microscopic approach.

The discretised Boltzmann equation is referred to as the
lattice-Boltzmann equation and is one of the fundamental corner stones
in the lattice-Boltzmann method, it reads:

\begin{equation}\label{eq:lbm:lbe}
f_i(\x + \cbf_i\delta_t, t + \delta_t) - f_i(\x, t) = \Omega_i(\x, t)
\end{equation}
where $f_i$ denotes the distribution function for direction $\cbf_i$,
$\delta_t$ is the time step and $\Omega_i$ is the (for
now non-specified) collision operator. Various forms of collision
operators exist and will be further discussed in section
\ref{sec:lbm:col}.

\subsection{Computational algorithm}
When solving eq. \eqref{eq:lbm:lbe} two computational tasks is
unidentified. Thus  

\section{The BGK collision operator}\label{sec:lbm:col}
The collision term in the LBE is the main ingredient in what
determines the physics of the system that is being modelled. Here the
desired interaction of the pseudo particles is stated. In section
\ref{sec:lbm:stat}, two necessary properties to approximations of the
full collision integral was stated.

One of the simplest collision operators that fulfil conditions (1)
and (2) in section \ref{sec:lbm:stat} is the BGK operator (BGK from
its creators: Bhatnagar, Gross and Krook). It was proposed in 1954 and
is today one of the most commonly used collision operators both in the
case of the lattice-Boltzmann and the continuous Boltzmann
equation. It is based on the principle of relaxing $f$ towards a
Maxwellian distribution. The relaxation is also performed in such a
way that the collision invariants are preserved.  In the discrete case,
eq. \eqref{eq:lbm:lbe} the operator is given by:

\begin{equation}\label{eq:lbm:bgk}
\Omega_{ij} = \Omega_i = -\omega \left[ f_i(\x, t) - f_i^{(eq)}(\x, t)
  \right]
\end{equation}
where $\omega$ is a parameter determining the relaxation rate and
$\feq$ should be an equilibrium distribution that makes sure that the
necessary conditions are fulfilled. In the discrete case, a truncated
expansion of eq. \eqref{eq:lbm:maxwell} is typically used
\cite{wolf-gladrow}. This gives for instance


\begin{equation}
\feq = w_i\rho \left [ 1 + \frac{\ci \cdot \ubf}{c_s^2} +
  \frac{(\ci \cdot \ubf)^2}{2c_s^4} - \frac{\ubf^2}{2c_s^2} \right]
\end{equation}
where $w_i$ is a lattice specific weight, $\rho$ is the zeroth moment
of $\fii$ and $\ci$ is a unit velocity in the discretised velocity
space.

The BGK operator is due to its simplicity both when it comes to
theoretical treatment and implementation a popular choice. However in
some physical situations, e.g. multi-phase or high Reynolds-number
flows, more sophisticated alternatives are required
\cite{wolf-gladrow}. Throughout this work, the BGK operator will be
used.

\nomenclature{BGK}{Relaxation type collision operator (Bhatnagar,
  Gross and Krook)}


\section{The lattice}\label{sec:lbm:lattice}
symmetry... multi speed... weights ... D2Q9
\begin{equation}\label{eq:lbm:weights}
w_i = ...
\end{equation}

\section{Asymptotic analysis}\label{sec:lbm:asym}
Methods from asymptotic analysis will, in this section, be used to
investigate the macroscopic limit of the general LBE. Detailed and
specific analyses for the three different equations considered will be
presented in sections \ref{sec:lbm:asym_np}, \ref{sec:lbm:asym_ns} and
\ref{sec:lbm:asym_pe}. Asymptotic analysis is basically about
describing mathematical objects in some limit, e.g. how a function
behaves for large or small values of some parameter. Consider for
example the series expansion of $\exp(\epsilon)$:

\begin{equation}
\exp(\epsilon) = 1 + \epsilon + \epsilon^2/2 + \epsilon^3/6 + \mathcal{O}(\epsilon^4) 
\end{equation}
It is clear that for sufficiently small values of $\epsilon$, the
terms of higher order is negligible to those of lower order and the
series may be truncated at some point and still be a good
approximation of the expression.

There are different approaches to go from the discrete LBE to a
continuous macroscopic equation. The most frequently applied one to
obtain the Navier-Stokes equations is the Chapman-Enskog method
\cite{junk-boundary}, which will reproduce the compressible
equations. An other method, often employed by M. Junk and his
associates e.g. in \cite{junk-asym}, is a method based on regular
asymptotic expansions, this is the method that will be utilised in
this work and will in the case of Navier-Stokes reproduce the
incompressible equations. A brief discussion of the differences between
the Chapman-Enskog and the regular expansion approaches will be carried
out at the end of this section.

The basic idea behind the analysis is to expand the distribution
function $\fii$ in some small parameter, $\epsilon$. Also this
parameter will be related to the spatial and time scales. The
macroscopic limit is obtained by taking the taylor expansion of the
discrete LBE and comparing terms of equal order in
$\epsilon$. Together with the fact that certain quantities is
invariant under collisions, macroscopic differential equations is
obtained. Now follows the part of the analysis which is common for the
three equations, the more equation specific analysis is carried out in
sections \ref{sec:lbm:asym_np}, \ref{sec:lbm:asym_ns} and
\ref{sec:lbm:asym_pe} respectively.

\subsection{Motivation of the choice of expansion parameter}
The expansion parameter should be a small dimensionless number. If the
lattice is dense enough with respect to the characteristic length
scale of the system, a suitable choice is the Knudsen number which is
defined as the ratio of the mean free path, $\delta_x$, and the
characteristic length of the system under consideration, $\ell_0$,
i.e. $\epsilon = \delta_x /\ell_o$. To be able to perform the
asymptotic analysis we must also relate the time scale to this
parameter. From the fact that the lattice speed $c_s =
\delta_x/\delta_t$ and by introducing a characteristic speed, $u_o =
\ell_0/t_0$, we have

\begin{equation}\label{eq:lbm:rel}
\epsilon = \frac{\delta_x}{\ell_0} = \frac{c_s}{u_0}\frac{\delta_t}{t_0}
\end{equation}
It is now clear that what determines the relation between the
timescale and the parameter $\epsilon$ is the ratio of the
characteristic speed and the lattice speed which is usually referred
to as the Mach number, Ma. In our particular case we will operate in
the incompressible limit, i.e. Ma $\ll$ 1 and a suitable choice is a
small number, thus Ma = $\epsilon$ is chosen \cite{junk-boundary}. The
discretisation of the space and time step is then

\begin{equation}
\delta_x'^2 = \delta_t' = \epsilon^2
\end{equation}
where the primes denote dimensionless variables. This particular
scaling is usually referred to as diffusive scaling.

\subsection{Expanding the LBE}
The LBE, eq. \eqref{eq:lbm:lbe}, with dimensionless variables and the
BGK collision operator reads:

\begin{equation}\label{eq:lbm:nodim_lbe}
f_i(\x' + \epsilon \cbf_i', t' + \epsilon^2) - f_i(\x', t') = -\omega \left[
  f_i(\x', t') - f_i^{(eq)}(\x', t') \right].
\end{equation} 
The primes denoting dimensionless variables will, for readability
reasons, from hereon be dropped. If nothing else is stated we always
consider dimensionless variables.

To obtain a differential equation, the difference equation in
eq. \eqref{eq:lbm:nodim_lbe} is taylor expanded, which gives

\begin{equation}\label{eq:lbm:taylor_lbe}
\ep(\pd\fii) + \ep^2 (\partial_t\fii + (\pd \fii)^2/2 ) + \ep^3
(\partial_t (\pd\fii) + (\pd\fii)^3/6) + \bigO{\ep^4} = 
-\omega \left[
  f_i - f_i^{(eq)} \right]
\end{equation}
Expanding also $\fii$ and $\feq$ in the parameter $\epsilon$:

\begin{equation}\label{eq:lbm:fi_exp}
\fii = \fie{0} + \ep\fie{1} + \ep^2\fie{2} + \ep^3\fie{3} + \bigO{\ep^4}
\end{equation}

\begin{equation}\label{eq:lbm:fi_eq_exp}
\feq = \feqe{0} + \ep\feqe{1} + \ep^2\feqe{2} + \ep^3\feqe{3} +
\bigO{\ep^4}
\end{equation}
and inserting these expressions into eq. \eqref{eq:lbm:taylor_lbe}
gives an equation with terms of varying orders of $\ep$. Separating
this equation in equations of common orders allows for an analysis of
what happens at different scales of $\epsilon$. For the four leading
orders in $\ep$ we have:

\begin{equation}\label{eq:lbm:ep0}
\ep^0:\;\; 0 = -\omega \left[
  \fie{0} - \feqe{0} \right],
\end{equation}

\begin{equation}\label{eq:lbm:ep1}
\ep^1:\;\; \pd\fie{0} = -\omega \left[
  \fie{1} - \feqe{1} \right],
\end{equation}

\begin{equation}\label{eq:lbm:ep2}
\ep^2:\;\; \pd\fie{1} + \partial_t\fie{0} + (\pd \fie{0})^2/2 =
-\omega \left[ \fie{2} - \feqe{2} \right]
\end{equation}
and
\begin{equation}\label{eq:lbm:ep3}
\ep^3:\;\; \pd\fie{2} + \partial_t\fie{1} + (\pd \fie{1})^2/2 +
\partial_t (\pd\fie{0}) + (\pd\fie{0})^3/6 = -\omega \left[ \fie{3} -
  \feqe{3} \right].
\end{equation}

The idea is now that for an equation of a particular order in $\ep$,
use collision invariants and eliminate unknown $\fie{n}$:s by using
equations of lower order in $\ep$. This will in the end result in
differential equations of macroscopic variables, given by moments of
the $\fii$:s.


\section{LBM for the Nernst-Planck equation}\label{sec:lbm:np}
The method presented here is based on representing the Nernst-Planck
equation, eq. \eqref{eq:et:np}, as an equation of advection-diffusion
type. Considering the quantity:  

\begin{equation}\label{eq:lbm:eff_adv}
\bar{\ubf} = \ubf -
  \frac{zq_eD}{k_BT}\nabla\psirm
\end{equation}
as an effective advective velocity, we have:

\begin{equation}\label{eq:lbm:adv-dif}
\dfrac{\partial \rho}{\partial t} + \nabla \cdot ( \bar{\ubf} \rho -
  D\nabla \rho ) = 0
\end{equation}
which is a mass conservation equation with fluxes from diffusion and
from advection respectively. The letter $\C$ for denoting the charge
concentration has in this section been replaced by the letter $\rho$
to avoid the risk of confusing it with the lattice velocities
which traditionally are denoted by ${\ci}$.

A collision operator of BGK type, eq. \eqref{eq:lbm:bgk} will be used
together with a D2Q9 lattice. The lattice-Boltzmann equation then
reads:

\begin{equation}
f_i(\x + \cbf_i\delta_t, t + \delta_t) - f_i(\x, t) = -\omega \left[ f_i(\x, t) - f_i^{(eq)}(\x, t) \right]
\end{equation}
with $\{\cbf_i\}_{i=0}^{Q-1}$ for the D2Q9 lattice as in
eq. \eqref{eq:lbm:d2q9_c}. The equilibrium function, $\feq$, is chosen
as \cite{alexey-tobias}:

\begin{equation}\label{eq:lbm:np_feq}
\feq = w_i \rho \left ( 1 + \frac{\ci \cdot \ubar}{c_s^2} \right)
\end{equation}
with the weights, $w_i$, as in eq. \eqref{eq:lbm:weights}. The charge
density and charge flux density is obtained by taking the zeroth and
first moments of the distribution function respectively, i.e:

\begin{equation}\label{eq:lbm:rho_mom}
\rhorm = \sum_i \fii
\end{equation}
and
\begin{equation}\label{eq:lbm:j_mom}
\jj = \sum_i \fii \ci
\end{equation}
The diffusion constant, $D$, is related to the relaxation parameter
$\omega$ through 

\begin{equation}\label{eq:lbm:np_D}
D = c_s^2 \left( \frac{1}{\omega} - \frac{1}{2} \right).
\end{equation}

\subsection{Asymptotic analysis}\label{sec:lbm:asym_np}
To motivate the appearance of the above suggested method for solving
eq. \eqref{eq:lbm:adv-dif} and for showing under what premises the
method is valid, the macroscopic limit of the discrete scheme will now
be analysed in an asymptotic manner. Note that for the advection
diffusion equation mass is but flux is not conserved.

From the expansion of $\fii$ in eq. \eqref{eq:lbm:fi_exp} and from
eqs. \eqref{eq:lbm:rho_mom} and \eqref{eq:lbm:j_mom} follow the
expansions of the mass density and flux respectively as

\begin{equation}\label{eq:lbm:rho_exp}
\rho = \rhoe{0} + \ep\rhoe{1} + \ep^2\rhoe{2} + \ep^3\rhoe{3} + \bigO{\ep^4}
\end{equation} 
and

\begin{equation}
\jj = \je{0} + \ep\je{1} + \ep^2\je{2} + \ep^3\je{3} + \bigO{\ep^4}
\end{equation} 
The advective velocity is also expanded as: 

\begin{equation}
\ubar = \ubare{0} + \ep\ubare{1} + \ep^2\ubare{2} + \ep^3\ubare{3} + \bigO{\ep^4}
\end{equation}
By plugging these expansion into the equilibrium distribution
eq. \eqref{eq:lbm:np_feq}, the expansion in
eq. \eqref{eq:lbm:fi_eq_exp} is obtained. The terms of order zero is
used in the zeroth order equation of the LBE, eq. \eqref{eq:lbm:ep0},
which gives

\begin{equation}
\fie{0} = w_i\rhoe{0} \left( 1 + \frac{\ci \cdot \ubare{0}}{c_s^2} \right).
\end{equation}
However, since we are only considering advection in the low Mach
limit, i.e. $|\ubar| \sim \ep$, we will in this analysis assume that
$\ubare{0} = 0$. $\ubar$ will then be of order $\ep$ to leading
order. It is possible to show \cite{junk-asym} that this assumption
holds if $\ubar$ is initialised properly, i.e. small and if no major
momentum sources are present. Thus the expression for $\fie{0}$
reduces to

\begin{equation}\label{eq:lbm:np_fi0}
\fie{0} = w_i\rhoe{0}.
\end{equation}

We now continue to the equation of order one in $\ep$,
eq. \eqref{eq:lbm:ep1}. Taking the zeroth moment gives the equation $0
= 0$ which indeed is true but not very useful. Note that the right
hand side vanishes due to mass conservation. $\fie{1}$ will be needed
in the next step and is, by using eq. \eqref{eq:lbm:np_fi0}:

\begin{equation}
\fie{1} = -\frac{1}{\omega} (\pd)(w_i\rhoe{0}) + w_i\left( \rhoe{1} +
\rhoe{0} \frac{\ci \cdot \ubare{1} }{c_s^2} \right).
\end{equation}
Taking the first moment of $\fie{1}$ gives the leading order in the
flux ($\je{0} = 0$ since $\ubare{0} = 0$):

\begin{equation}
\je{1} = \rhoe{0}\ubare{1} - c_s^2/\omega\nabla\rhoe{0}
\end{equation} 

Continuing to the equation of order two in $\ep$,
eq. \eqref{eq:lbm:ep2} and taking the zeroth moment of the equation
gives

\begin{equation}
\nabla \cdot \je{1} + \partial_t \rhoe{0} + c_s^2/2\nabla^2 \rhoe{0} = 0
\end{equation}
and by inserting the expression for $\je{1}$ we end up with

\begin{equation}
\partial_t \rhoe{0} + \nabla \cdot \left [ \rhoe{0}\ubare{1} - c_s^2
  \left( \frac{1}{\omega} - \frac{1}{2} \right)\nabla\rhoe{0} \right ]
= 0
\end{equation}
which is an advection diffusion equation with a diffusion constant as
in eq. \eqref{eq:lbm:np_D}. Since $\rhoe{0}$ fulfils the equation of
interest, the solution $\rho$ that we get from the lattice-Boltzmann
method is at least of first order accuracy. To determine the exact
order of accuracy, higher order terms, $\rhoe{k > 0}$, must be
determined. If also all those terms would be zero the method would be
exact, unfortunately that is not the case. The analysis of higher order
terms will not be performed here, but it is possible to show that
$\rhoe{1}$ is zero only under certain premises, i.e. for a proper
initialisation \cite{alexey-tobias}. $\rhoe{2}$ is however in general
non-zero and the obtained solution is thus second order accurate.  


\section{LBM for Navier-Stokes}
\subsection{Chapman-Enskog analysis}

\subsection{Forcing schemes}
how to add a ''forcing'' term in the method.

\section{LBM for Poisson's equation}
\subsection{Chapman-Enskog}

\section{Algorithm/Scheme for solving the coupled equations}
the iterative scheme used.

\section{Boundary conditions}
discussion and description of the boundary conditions
\subsection{bounce back}
accuracy, e.g. second order accurate if placed between node planes...
\subsection{slip}
\subsection{he-zou, constant density/velocity}
\subsection{Maybe something on non-local boundary conditions}

\section{Physical units vs lattice units}
how to interchange between them... etc...

\section{}
