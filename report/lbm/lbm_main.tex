\chapter{The lattice-Boltzmann method}
Rather than modelling on a macroscopic or microscopic scale, the
lattice-Boltzmann method operates at a scale in between those, often
referred to as a mesoscopic scale.

short intro of this section
micro macro meso... new method..

\section{Historical overview}
With the introduction of electronic calculating machines came also a
new possibility of tackling problems. New fields of computational
science was born and methods for solving both new traditional problems
were developed. 

The idea of using a discrete and simplified version of the
Boltzmann-equation dates back to the mid 60's \cite{scholarpedia-lbm}
with an experimental attempt to model simple gas dynamics. However, at
the time, this kind of statistical computational approaches was not
considered a serious alternative for the modelling of more
sophisticated fluid behaviour. It was first in the mid 80's when
Frisch, Hasslacher and Pomeau showed that a lattice automaton, with a
lattice of certain symmetry and that conserved mass and momentum in
the collisions, reproduced the Navier-Stokes equations in a
macroscopic limit. It was with this paper and the always increasing
computational power that made the idea of fluid modelling on a
mesoscopic scale a serious research topic.

The lattice gas automata (LGA) approach suffered from some flaws,
e.g. that the boolean nature of the method introduced statistical
noise and that lack of symmetry in the lattice made the advection
non-isotropic. The statistical noise was usually dealt with by
averaging which resulted in a coarsed domain and the advection issue
was handled by introducing lattices of higher symmetry. As the flaws
of the LGA approach was dealt with one after one, the method evolved
into what we today know as the lattice-Boltzmann method, with the
crucial refinement of using continuous distributions over boolean
variables. \cite{wolf-gladrow}

\section{The Boltzmann transport equation}
The continuous lbe is bte.. distribution function


\section{Basic idea of the LBM}
As previously noted, the lattice-Boltzmann is a mesoscopic
method. This means that the modelling is neither done on a molecular
level nor by direct solving the macroscopic equations involved. The
aim, in the most situations with the lattice-Boltzmann is indeed to
solve some macroscopic equation but not direct. Instead a statistical
model is used with various mesoscopic variables that, in some limit,
reproduces the macroscopic variables. It is also possible to ensure
that these variables fulfil a certain macroscopic equation by using a
certain scheme.

Basically the lattice-Boltzmann method solves a discretised version of
eq. \eqref{eq:lbm:boltzmann-eq} for the distribution functions and
then macroscopic quantities are determined from the distribution
functions. Both the spatial positions and the velocity space is
discretised allowing the distributions to ``sit'' only at certain
positions and to stream to neighbouring locations only in the
directions specified by the discretised velocity space. Usually in two
dimensions the velocity space is discretised into 9 distinct
velocities, more about the choice of lattice is discussed in section
\ref{sec:lbm:lattice}. In this case 9 distribution functions are
needed per node which might correspond to one or two macroscopic
variables but is indeed fewer than the number of variables needed for
a microscopic approach.

The discretised Boltzmann equation is referred to as the
lattice-Boltzmann equation and is one of the fundamental corner stones
in the lattice-Boltzmann method, it reads:

\begin{equation}\label{eq:lbm:lbe}
f_i(\x + \cbf_i\delta_t, t + \delta_t) - f_i(\x, t) = \Omega_i(\x, t)
\end{equation}
where $f_i$ denotes the distribution function for direction $\cbf_i$,
$\delta_t$ is the time step and $\Omega_i$ is the (for
now non-specified) collision operator. Various forms of collision
operators exist and will be further discussed in section
\ref{sec:lbm:col}.

\subsection{Computational algorithm}
When solving eq. \eqref{eq:lbm:lbe} two computational tasks is
unidentified. Thus  

\section{The collision operator}\label{sec:lbm:col}
discussions on different col. operators, focus on BGK since that is
the one used. collision invariants..

\subsection{The BGK operator}

\section{The lattice}\label{sec:lbm:lattice}
symmetry... multi speed... weights ... D2Q9

\section{LBM for Navier-Stokes}
\subsection{Chapman-Enskog analysis}

\subsection{Forcing schemes}
how to add a ''forcing'' term in the method.

\section{LBM for Poisson's equation}
\subsection{Chapman-Enskog}

\section{LBM for Nernst-Planck}
\subsection{Chapman-Enskog}

\section{Algorithm/Scheme for solving the coupled equations}
the iterative scheme used.

\section{Boundary conditions}
discussion and description of the boundary conditions
\subsection{bounce back}
accuracy, e.g. second order accurate if placed between node planes...
\subsection{slip}
\subsection{he-zou, constant density/velocity}
\subsection{Maybe something on non-local boundary conditions}

\section{Physical units vs lattice units}
how to interchange between them... etc...

\section{}
