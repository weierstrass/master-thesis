\section{Historical overview}
With the introduction of electronic calculating machines came also a
completely new possibility of tackling problems. New fields of
computational science was born and methods for solving both new and
traditional problems were developed.

The idea of using a discrete and simplified version of the
Boltzmann-equation dates back to the mid 60's \cite{scholarpedia-lbm}
with an experimental attempt to model simple gas dynamics. However, at
the time, this kind of statistical computational approaches was not
considered a serious alternative for the modelling of more
sophisticated and complex systems such as fluid behaviour. It was
first in the mid 80's when Frisch, Hasslacher and Pomeau showed that a
lattice automaton that conserved mass and momentum in the collisions
and with a lattice of certain symmetry, reproduced the Navier-Stokes
equations in a macroscopic limit. It was by their work and the always
increasing computational power that made the idea of fluid modelling
on a mesoscopic scale a serious research topic. \cite{wolf-gladrow}

The lattice gas automata (LGA) approach was not perfect and suffered
from some notable flaws, e.g. that the boolean nature of the method
introduced statistical noise and that lack of symmetry in the lattices
used made the advection non-isotropic. The statistical noise was
usually dealt with by averaging which resulted in a coarsed domain and
the advection issue was handled by introducing lattices of higher
symmetry. An other consequence of the boolean variables is that only one
particle per state was allowed which resulted in an equilibrium state
from Fermi-Dirac statistics rather than the desired Maxwell-Boltzmann
statistics. 

As the flaws of the LGA approach was resolved one by
another, the method evolved into what we today know as the
lattice-Boltzmann method, with the crucial refinement of using
continuous distributions over boolean variables. \cite{wolf-gladrow}

Today, the lattice-Boltzmann method is in many situations indeed a
competitor to more traditional CFD methods. For example with
advantages when it comes to parallelisation or implementing boundary
conditions in complex geometries. One major downside with the LBM is
the lack of theoretical work done and lack of literature compared to
the case with more traditional methods such as finite element/volume
methods. \cite{junk-asym}

\nomenclature{LGA}{Lattice Gas Automata/Automaton}
\nomenclature{CFD}{Computational Fluid Dynamics}
