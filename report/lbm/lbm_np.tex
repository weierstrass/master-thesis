\section{LBM for the Nernst-Planck equation} 
The method presented here is based on representing the Nernst-Planck
equation, eq. \eqref{eq:et:np}, as an equation of advection-diffusion
type. Considering the quantity:  

\begin{equation}
\bar{\ubf} = \ubf -
  \frac{zq_eD}{k_BT}\nabla\psirm
\end{equation}
as an effective advective velocity, we have:

\begin{equation}\label{eq:lbm:adv-dif}
\dfrac{\partial \rho}{\partial t} + \nabla \cdot ( \bar{\ubf} \rho -
  D\nabla \rho ) = 0
\end{equation}
which is a mass conservation equation with fluxes from diffusion and
from advection respectively. The letter $\C$ for denoting the charge
concentration has in this section been replaced by the letter $\rho$
for avoiding the risk of confusing it with the lattice velocities
which traditionally are denoted by ${\ci}$.

A collision operator of BGK type, eq. \eqref{eq:lbm:bgk} will be used
together with a D2Q9 lattice. The lattice-Boltzmann equation then
reads:

\begin{equation}
f_i(\x + \cbf_i\delta_t, t + \delta_t) - f_i(\x, t) = -\omega \left[ f_i(\x, t) - f_i^{(eq)}(\x, t) \right]
\end{equation}
with $\{\cbf_i\}$ for the D2Q9 lattice as in
eq. \eqref{eq:lbm:d2q9_c}. The equilibrium function, $\feq$, is chosen
as \cite{alexey-tobias}:

\begin{equation}\label{eq:lbm:np_feq}
\feq = w_i \rho \left ( 1 + \frac{\ci \cdot \ubar}{c_s^2} \right)
\end{equation}
with the weights, $w_i$, as in eq. \eqref{eq:lbm:weights}. The charge
density and charge flux density is obtained by taking the zeroth and
first moments of the distribution function respectively, i.e:

\begin{equation}\label{eq:lbm:rho_mom}
\rho = \sum_i \fii
\end{equation}
and
\begin{equation}\label{eq:lbm:j_mom}
\jj = \sum_i \fii \ci
\end{equation}
The diffusion constant, $D$, is related to the relaxation parameter
$\omega$ through 

\begin{equation}
D = c_s^2 \left( \frac{1}{2} - \frac{1}{\omega} \right).
\end{equation}

\subsection{Asymptotic analysis}
To motivate the appearance of the above suggested method for solving
eq. \eqref{eq:lbm:adv-dif} and for showing under what premises the
method is valid, the macroscopic limit of the discrete scheme will
now be analysed in an asymptotic manner.

From the expansion of $\fii$ in eq. \eqref{eq:lbm:fi_exp} and from
eqs. \eqref{eq:lbm:rho_mom} and \eqref{eq:lbm:j_mom} follow the
expansions of the charge and flux density respectively as

\begin{equation}
\rho = \rhoe{0} + \ep\rhoe{1} + \ep^2\rhoe{2} + \ep^3\rhoe{3} + \bigO{\ep^4}
\end{equation} 
and

\begin{equation}
\jj = \je{0} + \ep\je{1} + \ep^2\je{2} + \ep^3\je{3} + \bigO{\ep^4}
\end{equation} 
The advective velocity is also expanded as: 

\begin{equation}
\ubar = \ubare{0} + \ep\ubare{1} + \ep^2\ubare{2} + \ep^3\ubare{3} + \bigO{\ep^4}
\end{equation}
By plugging these expansion into the equilibrium distribution
eq. \eqref{eq:lbm:np_feq}, the expansion in
eq. \eqref{eq:lbm:fi_eq_exp} is obtained. The terms of order zero is
used in the zeroth order equation of the LBE, eq. \eqref{eq:lbm:ep0},
which gives

\begin{equation}
\fie{0} = w_i\rhoe{0} \left( 1 + \frac{\ci \cdot \ubare{0}}{c_s^2} \right).
\end{equation}
However, since we are only considering the low Mach limit,
i.e. $|\ubar| \sim \ep$, we will in this analysis assume that
$\ubare{0} = 0$. And thus $\ubar$ will be of order $\ep$ to leading
order. It is possible to show \cite{junk-asymp} that if $\ubar$ is
initialised small it will also stay small if no major momentum sources
are present, thus it is a question of proper initialisation if the
assumption holds or not. Thus the expression for $\fie{0}$ reduces to

\begin{equation}
\fie{0} = w_i\rhoe{0}
\end{equation}


