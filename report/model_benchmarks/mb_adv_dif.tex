\section{Advection-Diffusion}
Before the implementation of the Nernst-Planck part of the model is
tested, a special case is considered, i.e. when the electrical
potential in the domain is constant. This makes the flux term
including the electrical potential in eq. \eqref{eq:et:np} vanish
and we have to solve only for pure advection and diffusion.

Introducing characteristic scales for the concentration ($\C_0$),
advective velocity ($u_0$) and length ($l_0$) respectively, gives the
non-dimensional advection-diffusion equation:

\begin{equation}\label{non_dim_adv-dif}
\frac{\partial \C}{\partial t} + \mathbf{u} \cdot \nabla \C -
\frac{D}{u_0 l_0} \nabla^2\C = 0.
\end{equation}

All variables in \eqref{non_dim_adv-dif} are non-dimensional. The
quantity $\Pe = u_0l_0/D$ is often referred to as the P\'{e}clet number. It
determines the relation between contributions to the
dynamics from advection and diffusion respectively. For $\Pe \gg 1$ the
dynamics is dominated by advection and for $\Pe \ll 1$ by diffusion. 

The LB model described in section \ref{sec:lbm:np} was tested by
studying the evolution in time and space of a point mass in one
dimension. The analytical solution of eq. \eqref{non_dim_adv-dif} in
one dimension with initial conditions $\C(x, t = 0) = \delta(x)$ on an
infinite domain is:

\begin{equation}
\C(x, t) = \sqrt{\frac{\Pe}{4 \pi t}}\exp\left({-\frac{(x - ut)^2
    \Pe}{4t}}\right).
\end{equation}

In the numerical computation, the parameters $\Pe = 10$ and
$|\mathbf{u}| = 0.1$ were used. The domain consists of 200 lattice
nodes and three snapshots in time at $t = 100, 200, 300$ l.u. were
compared to the analytical solution. The result is presented in
fig. \ref{fig:adv-dif}.

\begin{figure}
\begin{center}
\includegraphics[width=0.9\textwidth]{fig/adv_dif_10_30_50.pdf}
\end{center}
\caption[Computed solutions of the advection-diffusion equation.]{Computed
  solutions ($\times$) of the advection-diffusion equation for a point
  mass evolving in time and space. Three different times ($t_n =
  100n$) are compared to analytical solutions (solid). The Amplitude
  of the solutions as function of time has also been plotted
  (dashed). The advecting velocity, $u_0 = 0.1$ was used together with
  a Peclet number, $\Pe = 10$. All units are in lattice units.}
\label{fig:adv-dif}
\end{figure}
