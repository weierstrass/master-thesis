\chapter{Code snippet}\label{app:poi_snippet}
An example code defining Poiseuille flow driven by a constant force
(pressure gradient) using the LBM library developed in this work.
\begin{verbatim}
#include <iostream>
#include "../LBM.h"

using namespace std;

int main(){
  int nx = 3, ny = 50, tMax = 1000;
  double w = 0.75;
  double c = 1.0;

  LatticeModel *lm = new Lattice2D(nx, ny);
  StreamD2Q9Periodic *sm = new StreamD2Q9Periodic();
  CollisionD2Q9BGKNSF *cm = new CollisionD2Q9BGKNSF();
  LBM *lbm = new LBM(lm, cm, sm);

  double **fx = allocate2DArray(ny, nx);
  double **fy = allocate2DArray(ny, nx);

  cm->setW(w);
  cm->setC(c);


  /* Set boundary conditions*/
  BounceBackNodes<CollisionD2Q9BGKNSF> *bbns =
          new BounceBackNodes<CollisionD2Q9BGKNSF>();
  bbns->setCollisionModel(cm);
  for(int i = 0; i < nx; i++){
    bbns->addNode(i, 0, 0);
    bbns->addNode(i, ny-1, 0);
  }
  lbm->addBoundaryNodes(bbns);

  /* Set force */
  for(int i = 0; i < nx; i++){
    for(int j = 0; j < ny; j++){
      fx[j][i] = 0.0001;
      fy[j][i] = 0.0;
    }
  }
  cm->setForce(fx, fy);

  /* Initialize solver */
  lbm->init();

  /* Main loop */
  for(int t = 0; t < tMax; t++){
    lbm->collideAndStream();
  }

  cm->dataToFile("bench_force_poi/");

  return 0;
}
\end{verbatim}
