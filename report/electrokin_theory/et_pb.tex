\section{Poisson-Boltzmann equation}
A simple and commonly used approach for determining potentials (and
charge distributions) in systems with present EDLs is by solving
eq. \eqref{eq:pb} with a charge distribution of Boltzmann
type. Here follows a brief derivation of this term together with some
discussion on the assumptions made.

The fundamental assumption that the derivation of the charge
distribution is based on, is the fact that the system is assumed to be
under thermodynamical equilibrium. I.e. forces, acting on the ions,
due to chemical diffusion from concentration gradients and from the
electrical field are therefore balancing each other. In one dimension:

\begin{equation}\label{eq:dif_elec_forces}
\frac{d \mu_i}{dx} = -z_i q_e\frac{d\psi}{dx}
\end{equation}
where $\mu_i$ is the chemical potential for species $i$, $z_i$ is the
relative charge of species $i$, $q_e$ the fundamental charge and
$\psi$ is the EDL potential. The chemical potential is given by
\cite{ren}:

\begin{equation}
\mu_i = \mu_i^{\infty} + k_BT\ln n_i
\end{equation}
where $\mu_i^{\infty}$ is a reference value for the chemical potential,
here the potential value far from the charged wall is used, $k_BT$ is
the thermal energy and $n_i$ is the ion concentration of species
$i$. This expression plugged into eq. \eqref{eq:dif_elec_forces} gives

\begin{equation}\label{eq:eq_for_ni}
\frac{d \ln(n_i)}{dx} = - \frac{z_i q_e}{k_BT}\frac{d \psi}{dx}.
\end{equation}
The charge density is determined by solving eq. \eqref{eq:eq_for_ni}
for $n_i$, i.e. integrating the equation. In order to avoid
introducing additional unknown quantities, the equation is integrated
to far away from the wall where the potential from the EDL has
decreased to zero and where the concentrations, $n_i^{\infty}$,
of the ions are known.

\begin{equation}
\int_x^{\infty} d\ln( ni(x')) = -\frac{z q_e}{k_BT}\int_x^{\infty}d\psi(x')
\end{equation}
This finally gives an expression for $n_i(x)$:

\begin{equation}\label{eq:ni}
n_i(x) = n_i^{\infty} \exp\left(-\frac{z_i q_e \psi(x)}{k_BT}\right).
\end{equation}
and $\rho_e$ is given by

\begin{equation}\label{eq:rho}
\rho_e = q_e\sum_i z_i n_i.
\end{equation}

Summarising eqs. \eqref{eq:pb}, \eqref{eq:ni} and \eqref{eq:rho} gives
the Poisson-Boltzmann equation in one dimension

\begin{equation}\label{eq:pb_real}
\frac{d^2\psi(x)}{dx^2} = -\frac{q_e}{\epsilon_r \epsilon_0}\sum_i z_i
n_i^{\infty} \exp\left(-\frac{z_i q_e \psi(x)}{k_BT}\right).
\end{equation}

\subsection{The Debye–Hückel approximation}
In the early days of the study of ion solutions the non-linear nature
of eq. \eqref{eq:pb_real} complicated for the ones trying to solve
it. A linearisation was usually the way to go.

\subsection{Limitations of the Poisson-Boltzmann model}
From the derivation of the Poisson-Boltzmann equation above, two main
assumptions are made. First the system is considered to be at thermal
equilibrium. And second the system is assumed to infinitely large. 

In this work, flows of ionic solution will be studied and the
assumption with thermodynamical equilibrium does not apply. However
for low-speed flows the model may still be a decent approximation,
which will be investigated. 

The second assumption, may also stay unfulfilled in some cases
investigated here. The fluid in contact with the wall must be of
substantial size in relation to the EDL thickness. There will be
cases where the choice of $\zeta$ potential in combination with thin
channels will make this assumption not fulfilled.

Since the PB equation is unable to model the system of
interest, a different approach will be presented. However, throughout
this work, references and comparisons with the PB model will be made. 


%intro

%liten härledning av högerledet 
%diffusion by conc grad. = grad of
%potential <==> termodynamic equilibrium
%chem pot definerad som....
% eq.
% boundary conditions
%antagaganden !!!
