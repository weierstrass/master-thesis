\section{Pressure-driven electrokinetic flow}\label{sec:et:streaming_pot}
As a charged fluid is driven by a pressure gradient, a movement of
charges, i.e. an electrical current will be induced. Due to the charge
flux, a potential gradient will build up along the flow
direction. This potential is usually referred to as the streaming
potential, $\phi(\x)$, and its magnitude is determined from the
induced current through Ohm's law

\begin{equation}
\J = -  \sigma \nabla \phi  
\end{equation}   
where $\sigma$ is the conductivity of the fluid. For a perfectly
conducting fluid, no potential difference will be built up. Also a
complete neutral solution will carry no net current and therefore will
no potential difference build up in that case either.

Charges under the influence of an electric field will be affected by a
force. Charges moving due to this force will, in a liquid, also pull
liquid (uncharged) molecules with them. In a macroscopic limit, the
force density affecting the charges in the liquid is assumed to affect
the liquid as whole. The volumetric force affecting the fluid from
the presence of the streaming potential is then given by:

\begin{equation}
\F = \rho_e \nabla \phi
\end{equation}
where $\rho_e$ is the charge density. This is an example of how the
charge density from the Nernst-Planck equation may couple to the force
term in Navier-Stokes equations. 

This force will always be affecting the fluid in an direction opposite
of that of the net flux of charge, i.e. the force will slow the fluid
down, this is illustrated in fig. \ref{fig:et:electrovis}. This effect
that a moving net charged fluid is slowed down is called the
\emph{electroviscous effect}. The name originates from that a similar
effect might be achieved by increasing the viscosity of the fluid.

\todo{bild ion channel!!}
