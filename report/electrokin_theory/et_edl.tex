\section{Basic concepts of electrokinetic flow}
Electrohydrodynamics involves the study of electric phenomena on fluid
flow. How fluids carrying electrical charges (electrolytes) react upon
external electrical fields or interact with charged objects are
examples of problems that arise in this field. 

\subsection{Electrical double layers}
As a charged object is brought into contact with an electrolyte it is,
qualitatively, easily deduced that ions with a sign of charge opposite
to that of the object will be attracted to the object and ions with
the same sign of charge will be repelled. These two distinct
categories of ions will from hereon be referred to as counter- and
co-ions respectively. In this case, for a neutral electrolyte, a
surplus of counter-ions will be present in the direct vicinity of the
object and a surplus of co-ions will be present at some other location
further from the object.
 
The area with a surplus of counter-ions in an electrolyte in contact
with a charged object is often referred to as an electrical double
layer. Two distinct regions will be formed in this area, thus the name 
double layer. The two layers are often referred to as the Stern
layer (adsorbed ions) and the diffusive layer (mobile ions). The Stern
layer is usually several orders of magnitude thinner than the
diffusive layer and is therefore seldom considered when it comes to
modelling \cite{donquing-ren}. 

\subsection{Electroosmosis}
As fluid carrying a net charge, e.g. in the diffusive layer of an EDL,
is under influence of an electric field, the charged particles will
move due to the electric forces. As the charge particles move, they
will affect the surrounding liquid, causing it move as well. This
liquid motion is often referred to as electroosmotic
flow. \cite{dongquing-ren-book}

%% Consider an electrically neutral liquid, i.e. a liquid containing the
%% same amount of positive and negative ions. When this liquid is
%% introduced to, for example, a negatively charged surface, this even
%% charge distribution is disturbed in an area close to the surface. Due
%% to the introduced electrostatic forces, positive ions will be
%% attracted to the surface leaving a positive net charge in the vicinity
%% of the surface. It is possible to divide this positively charged
%% region in the liquid into two different layers. In the direct vicinity
%% of the surface, positive ions will adsorb onto the surface making them
%% less mobile than the others in the positively net charged area closer
%% to the bulk liquid.  This is also
%% illustrated in fig. \ref{fig:edl_charge} \cite{ren_book}

%% %potential charge general
%% The interface between the Stern and the diffusive layer is often
%% called the shear plane. Due to the difficulty of measuring the
%% potential at the true surface, i.e. the one in contact with the Stern
%% layer of the liquid, most models in the field of electrokinetics use
%% the shear plane as the boundary for which it exists accurate methods
%% to measure the potential \cite{ren_book}. The potential at the shear
%% plane will, from heron, be referred to as the $\zeta$-potential.



%poisson boltzmann.
