\section{Electrical double layers}
Consider an electrically neutral liquid, i.e. a liquid containing the
same amount of positive and negative ions. When this liquid is
introduced to, for example, a negatively charged surface, this even
charge distribution is disturbed in an area close to the surface. Due
to the introduced electrostatic forces, positive ions will be
attracted to the surface leaving a positive net charge in the vicinity
of the surface. There is possible to divide this positively charged
region in the liquid into two different layers. In the direct vicinity
of the surface, positive ions will adsorb onto the surface making them
less mobile than the others in the positively net charged area closer
to the bulk liquid. The two layers are often referred to as the Stern
layer (adsorbed) and the diffusive layer (mobile). This is also
illustrated in fig. \ref{fig:edl_charge} \cite{ren_book}

%potential charge general
The interface between the Stern and the diffusive layer is often
called the shear plane. Due to the difficulty of measuring the
potential at the true surface, i.e. the one in contact with the Stern
layer, most models in the field of electrokinetics use the shear plane
as the boundary for which it exists accurate methods to measure the
potential \cite{ren_book}. The potential at the shear plane will,
from heron, be referred to as the $\zeta$-potential. 

To be able to model the flow dynamics of liquids in channels with
present EDLs, the potential and charge distribution in the channel
must be determined. The potential and the charge distribution is
related through Poisson's equation for electrostatics:

\begin{equation}
\nabla^2\psi = \frac{\rho_e}{\epsilon_r \epsilon_0}
\end{equation}

where $\psi$ is the electrical potential, $\rho_e$ the electrical
charge density and $\epsilon_r \epsilon_0$ the absolute permittivity. 

Before the final model used in this project is
presented, a simpler approach based on the Poisson-Boltzmann equation
will be presented. 




%poisson boltzmann.
