\section{The transport of charges - Nernst-Planck equation}\label{sec:et:np}
The charge concentration in an electrolyte is indeed affected by its
environment. In the model proposed here, influences from: advection of
the electrolyte, diffusion due to concentration gradients and effects
from the electric field originating from charged objects placed at the
border or in the flow is considered. Charge conservation without any
external sources of the ion density, $\C(\x, t)$, gives:

\begin{equation}\label{eq:charge_conc}
\dfrac{\partial \C}{\partial t} + \nabla \cdot \J = 0
\end{equation}
where $\mathbf{J(\mathbf{x}, \mathrm{t})}$ is the net flux induced
by the effects described above. Explicit expressions for the fluxes
due to advection and diffusion respectively are 

\begin{equation}
\J_{adv} =
\C\ubf
\end{equation}
and 
\begin{equation}
\mathbf{J}_{dif} = -D\nabla \C 
\end{equation}
where $\mathbf{u}$ is the advective velocity and $D$ is a diffusion
coefficient. The ionic flux due to the presence of an electric
potential, $\psirm(\x, t)$, is given by the Nernst equation
\cite{dongquing-ren-book}:

\begin{equation}
\J_{ele} = -\frac{zq_eD}{k_BT}\C\nabla\psirm
\end{equation}
where $z$ is the relative charge of the ion species, $q_e$ is the
fundamental charge, $k_B$ is the Boltzmann constant and $T$ is the
temperature of the fluid.

Summing up the fluxes and putting them into eq. \eqref{eq:charge_conc}
gives

\begin{equation}\label{eq:et:np}
\dfrac{\partial \C}{\partial t} = \nabla \cdot \left [
 D\nabla \C - \C\ubf + \frac{zq_eD}{k_BT}\C\nabla\psirm
\right]
\end{equation}
which is a known result often referred to as the Nernst-Planck
equation. This is the equation for the transport of \emph{one} species
of ions, if several are present one NP equation for each species needs
to be solved. The advective velocity, $\ubf$, and the potential
gradient, $\nabla \psirm$, are obtained from couplings to the
Navier-Stokes and Poisson's equation respectively. More about the
coupling between the equations is discussed in section
\ref{sec:et:coupling}.

\subsection{Boundary conditions}
Depending on the physical situation being modelled, different
conditions may be imposed at the boundaries of the domain. Throughout
this work, at hard boundaries (walls), the charge flux through the
boundary is set to zero, i.e.:

\begin{equation}\label{eq:et:j0}
\J \cdot \mathbf{n} = 0 \;,\;\; \x \in \Gamma
\end{equation}
where $\mathbf{n}$ denotes the normal to the surface and $\Gamma$ is
the boundary of the domain. 


%what we need

%nernst planck


%poisson boltzmann steady state, fix potential 
