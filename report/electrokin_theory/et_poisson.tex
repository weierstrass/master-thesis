\section{The potential - Poisson's equation}
To be able to model the flow dynamics of liquids in a channel with
present EDLs, the potential and charge distribution
in the channel must be determined. These quantities are mutually
related through Poisson's equation for electrostatics:

\begin{equation}\label{eq:pb}
\nabla^2\psirm = -\frac{\rho_e}{\epsilon_r \epsilon_0}
\end{equation}

where $\psirm$ is the electrical potential, $\rho_e$ the electrical
charge density, $\epsilon_r$ is the relative permittivity and
$\epsilon_0$ the vacuum permittivity. Under certain assumptions, the
charge density may be explicitly determined as a function of the
potential distribution, one such result is the so called
Poisson-Boltzmann equation, further discussed in section \ref{sec:et:pb}.

\subsection{Boundary conditions}
At the charged boundaries, most physical situations may be covered by
either specifying the potential or the surface charge density. The
former would be boundary condition of Dirichlet type:

\begin{equation}
\psirm(\x) = \zeta(\x)\;,\;\; \x \in \Gamma
\end{equation}
and the latter a boundary condition of Neumann type:

\begin{equation}
\nabla\psirm(\x) \cdot \n =
-\frac{\sigma(\x)}{\epsilon_0\epsilon_r}\;,\;\; \x \in \Gamma
\end{equation}
where $\Gamma$ denotes the boundary of the domain and $\n$ is the
normal to the boundary surface.\- \cite{hlushkou}
