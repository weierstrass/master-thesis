\section{Electroosmotic flow}\label{sec:et:electroosmosis}
Instead of driving the fluid flow through a pressure drop, a net
charged fluid may be driven by an external electric field. This may
be seen as the opposite case to that in section
\ref{sec:et:streaming_pot} where a current is induced by a pressure
drop.

The volumetric force on the fluid from the external field, $\E_{ext}$,
is given by

\begin{equation}
\F = - \rhorm_E \E_{ext}
\end{equation}

where $\rho_e$ is the charge density. If the electric field is
constant (or at least has the same direction) everywhere, the sign of
the force is not in the same direction for a net charged positive area
of the fluid as for a net charged negative. Thus the fluid may be
either slowed down or sped up. This is a qualitative difference to
pressure driven situation and is illustrated in fig. \ref{fig:et:}.

The electroviscous effect is in the case of electroosmotic flow
usually neglected as the field due to the streaming potential is, in
most physical cases, small to the applied external
field. \cite{dongquing-ren} 

\todo{bild ion channel!}
