% Chalmers title page
\begin{titlepage}

\mbox{}
\vspace{50mm}
%draft text
\begin{center} 
\textcolor{red}{
\framebox[1.1\width]{
\Huge{DRAFT}
\huge{\today}
}
}
\end{center}

\AddToShipoutPicture{\backgroundpic{-4}{56.7}{fig/auxiliary/frontpage}}
\mbox{}
\vfill
\addtolength{\voffset}{2cm}
\begin{flushleft}
	{\noindent {\Huge Modelling of electrokinetic flow using the
            lattice-Boltzmann method} \\[0.5cm] \emph{\Large Thesis
            for the degree of Master of Science} \\[.8cm]
	
	{\huge ANDREAS B\"{U}LLING}\\[.8cm]
	
	{\Large Department of Mathematical Sciences \\
	\textsc{Chalmers University of Technology} \\
	Gothenburg, Sweden 2012 \\
	Master's Thesis 2012:1\\
	} 
	}
\end{flushleft}

\end{titlepage}
\ClearShipoutPicture
% End Chalmers title page

\pagestyle{empty}
\newpage
\clearpage
\mbox{}
\newpage
\clearpage
\thispagestyle{empty}

\begin{abstract}
The lattice-Boltzmann method is used to model flow in electrokinetic
systems. A modelling approach based on the coupling of Navier-Stokes,
Nernst-Planck and Poisson's equation of electrostatics is
utilised. Three lattice-Boltzmann methods are formulated for the three
equations respectively.

An implementation with the aim of being high performing is
done. Topics as locality, instruction pipelines and parallel computing
are considered. The implementation is tested for a number of classic
examples with known solutions, e.g.  Taylor vortex flow, an Helmholtz
equation and an advection-diffusion situation. The computed solutions
agree well with the analytic solutions.

The systems modelled consists mainly of various charged channel flows
of ionic solutions. Electrokinetic effects, such as electroosmosis and
the electrovicous effect are studied. This is done in thin channels
where the thickness of the electrical double layers is comparable to
the channel dimension. The electroviscous effect is shown to slow the
flow down and a local minimum is found in the velocity profile for
thick enough double layers. Other more complicated systems are also
studied; electroosmotic flow in a channel with heterogeneously charged
walls and flow in a an array of charged squares.

\end{abstract}

\newpage
\clearpage
\mbox{}
\newpage
\clearpage
\thispagestyle{empty}
\section*{Acknowledgements}
I would hereby like to express my gratitude to my supervisor Alexei
Heintz. You did always have time for questions and discussions even
during your spare time and on your vacation. Also the precise amount
of guidance and the freedom given is very much appreciated. I am
thankful for having been given the opportunity of working on a project
where the fields of my interest from physics, mathematics and
programming was all present.  \\[1cm]

\hfill Andreas B\"{u}lling, G\"{o}teborg, \today
\newpage
\clearpage
\mbox{}
