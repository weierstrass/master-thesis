\chapter{Introduction}
This thesis deals with modelling of physical problems in the
interdisciplinary field of hydrodynamics and electrostatics. The tool
used for realising this is the new and promising but somewhat immature
lattice-Boltzmann method. This is a method that is still under
development but is today used in practical applications both in
industry and academy.

\section{Background}

There is currently an ongoing project at the mathematics faculty of
Chalmers University in producing a modelling package that
should be able to deal with transport of various liquids and particles
through complicated structures. The method of choice has fallen upon
the lattice-Boltzmann method for its suitable characteristics in the
systems of interest. 

This work aims to investigate the possibility and procedure for taking
electrical effects into account in the modelling of charged
fluids. More theoretical questions about the method itself and of the
physics involved is of interest as well as how the method may be
effectively implemented on a computer.

From both industry and academy, there is a demand on the modelling of
this kind of physics. For instance, in medical sciences, accurate
modelling of transport of charged fluids is a fundamental ingredient
in understanding biological systems and to be able to manipulate
them. As a consequence of the always so present desire of more
environmental friendly ways of using the planet, automotive industry
are now engineering electrical cars. A great challenge is to produce
high performing and durable batteries, the ability to accurate
model the electrolytes in the batteries is indeed an advantage in
achieving this. 

\section{Outline}
The text is structured in five main chapters. In chapter \ref{sec:et},
is the physics involved and the equations of interest presented. This
is followed by chapter \ref{sec:lbm} where the lattice-Boltzmann
method is formulated for the different equations of interest. Also an
introduction to the method as well as some discussion on different
boundary conditions is given here. In chapter \ref{sec:hpc}, the
implementation of the method is discussed together with some general
aspects that is important to have in mind in order to produce a high
performing code. The implementation is then tested for classic
examples with known solutions in chapter \ref{sec:mb}. Finally some
results in electrokintetics are presented and discussed in chapter
\ref{sec:res}. 


